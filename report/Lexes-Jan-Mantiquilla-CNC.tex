\documentclass{article}

\usepackage{graphicx}
\usepackage{listings}
\usepackage{float}
\usepackage{caption}
\usepackage{hyperref}
\usepackage{amsmath}
\usepackage[square,numbers]{natbib}

%%------------------------------------------------
%% Image and Listing code
%%------------------------------------------------
%% Examples for the commands in the document below
%%
%% includecode:
%% \includecode{caption for table of listings}{caption for reader}{filename}
%% - includes a file with code and adds a caption that should describe the code in some detail and a shorter caption for the table of listings
\newcommand{\includecode}[4]{\lstinputlisting[float,floatplacement=H,
caption={[#1]#2}, captionpos=b, frame=single, label={#3}]{#4}}

%% includefigure:
%% \includefigure{label}{short caption}{long caption}{filename}
%% - includes a figure with a given label, a short caption for the table of contents and a longer caption that describes the figure in some detail
\newcommand{\includefigure}[4]{
  \begin{figure}[H]
    \centering
    \includegraphics{#4}
    \captionsetup{width=.8\linewidth}
    \caption[#2]{#3}
    \label{#1}
  \end{figure}
}

%% includescalefigure:
%% \includescalefigure{label}{short caption}{long caption}{scale}{filename}
%% - includes a figure with a given label, a short caption for the table of contents and a longer caption that describes the figure in some detail and a scale factor 'scale'
\newcommand{\includescalefigure}[5]{
  \begin{figure}[H]
    \centering
    \includegraphics[width=#4\linewidth]{#5}
    \captionsetup{width=.8\linewidth}
    \caption[#2]{#3}
    \label{#1}
  \end{figure}
}


%%------------------------------------------------
%% Parameters
%%------------------------------------------------
% Set up the header and footer
\newcommand{\assignmentTitle}{Assignment\ \#2: OpenFlow} % Assignment title
\newcommand{\moduleCode}{CS2031}
\newcommand{\moduleName}{Telecommunications\ II}
\newcommand{\authorName}{Lexes\ Jan\ Mantiquilla} % Your name
\newcommand{\authorID}{-1} % Your student ID


\title{\textbf{\moduleCode\ \moduleName\ \assignmentTitle}}
\author{\authorName\ -\ \authorID}
\bibliographystyle{abbrvnat}
% Fix url
\Urlmuskip=0mu plus 1mu
\graphicspath{{images/}}

%%------------------------------------------------
%% Document
%%------------------------------------------------
\begin{document}
\captionsetup{width=.8\linewidth}

\maketitle
\tableofcontents

\newpage

\section{Introduction}

\subsection{Problem Statement}

\subsection{Approach}

\section{Overall Design}

\subsection{Problem Statement}

\subsection{Features}

\subsection{OpenFlow}

\subsection{Design of the Protocol}

\subsection{Packet Descriptions}

\section{Implementation}
This section contains an explanation of the implementation. It will describe how
each of the classes used implement the design described above.

\subsection{Packet Classes}
The packet classes are encoded using the method ObejectInputStream and
ByteArrayInputStream. This allows us to create a byte array which is then used
to create a DatagramPacket. The abstract class PacketContent and the classes
which inherit it abstract the creation of DatagramPackets. The abstract
methods toByteArray and fromDatagramPacket do the encoding of the
DatagramPacket. Each implementation of the PacketContent class has their own
implementation of the abstract methods toByteArray and fromDatagramPacket. The
abstract PacketContent abstract class and the AckPacketContent class have been
provided to us in the Advanced sample code. I have created other packet classes
which implement the PacketContent abstract class.

\subsubsection{FeatureRequestPacketContent}
This class is used to create a packet which represents a feature request
packet. The feature request packet is sent by the \textbf{Controller} to a
switch upon receiving a Hello packet. The feature request packet does not
contain any data and only contains its type.

\subsubsection{FeatureResultPacketContent}
This class is used to create a packet which represents a feature result packet.
The feature result packet is sent by the \textbf{Switch} to the
\textbf{Controller} in response to a feature request packet from the
\textbf{Controller}. The feature request packet contains information about the
specifications of the \textbf{Switch} such as the number of tables it has and
the size of the buffer. The buffer size and number of tables currently have no
use and is only included as it was a part of the OpenFlow specification. The
feature result packet also contains the \textbf{Switch's} name and its
connections. The switch connections are used when calculating the path to the
specified destination.

\subsubsection{FlowModPacketContent}
This class is used to create a packet which represents a flow mod packet. The
flow mod packet is sent by the \textbf{Controller} to a \textbf{Switch}. The
flow mod packet is used to change a \textbf{Switch's} flow table. The flow mod
packet contains two strings, the destination and the next hop. Upon receiving a
flow mod packet, the \textbf{Switch} update it's flow table with the destination
as the key and the next hop as the value.

\subsubsection{HelloPacketContent}
This class is used to create a packet which represents a hello packet. The hello
packet is sent by both the \textbf{Switch} and \textbf{Controller}. The hello
packet contains the version number. When receiving a hello packet, the lower
version of the protocol is set. In my assignment, the default version number
is set to 1. In my version of OpenFlow, the hello packet is sent first by the
\textbf{Switch} to the \textbf{Controller} upon initialisation. The
\textbf{Controller} then responds with it's own hello packet back to the
\textbf{Switch}. This differs from the actual OpenFlow specification as the
hello packet should be asynchronous.

\subsubsection{PacketInPacketContent}
This class is used to create a packet which represents a packet in packet. The
packet in packet is sent by the \textbf{Switch} to the \textbf{Controller}.
This packet is used to ask the \textbf{Controller} for a flow mod packet. This
packet is sent in the case where a table miss occurs and the switch does not
know where to forward the packet. The packet in packet contains the final
destination of the packet and the switch name. The destination and the switch
name are both keys in the \textbf{Controller's} flow tables.

\subsubsection{PayloadPacketContent}
This class is used to create a packet which represents a payload packet. The
payload packet contains the payload i.e. the string containing the message and
final destination of the packet. This packet is sent by the \textbf{EndNodes}
as well as the \textbf{Switches}. This packet is the packet which is forwarded
from \textbf{Switch} to \textbf{Switch} and to the final destination
\textbf{EndNode}.

\subsection{Node Classes}

\subsubsection{Controller}

\subsubsection{EndNode}

\subsubsection{Node}

\subsubsection{Switch}

\subsection{Path Finding class}

\subsubsection{Graph}

\subsection{Miscellaneous Classes}

\subsubsection{Terminal}

\subsubsection{Tokenizer}

\section{Advantages and Disadvantages}

\subsection{Advantages}

\subsection{Disadvantages}

\section{Program Usage}

\section{Reflection}

\end{document}
