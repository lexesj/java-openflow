\documentclass{article}

\usepackage{graphicx}
\usepackage{listings}
\usepackage{float}
\usepackage{caption}
\usepackage{hyperref}
\usepackage{amsmath}
\usepackage[square,numbers]{natbib}

%%------------------------------------------------
%% Image and Listing code
%%------------------------------------------------
%% Examples for the commands in the document below
%%
%% includecode:
%% \includecode{caption for table of listings}{caption for reader}{filename}
%% - includes a file with code and adds a caption that should describe the code in some detail and a shorter caption for the table of listings
\newcommand{\includecode}[4]{\lstinputlisting[float,floatplacement=H,
caption={[#1]#2}, captionpos=b, frame=single, label={#3}]{#4}}

%% includefigure:
%% \includefigure{label}{short caption}{long caption}{filename}
%% - includes a figure with a given label, a short caption for the table of contents and a longer caption that describes the figure in some detail
\newcommand{\includefigure}[4]{
  \begin{figure}[H]
    \centering
    \includegraphics{#4}
    \captionsetup{width=.8\linewidth}
    \caption[#2]{#3}
    \label{#1}
  \end{figure}
}

%% includescalefigure:
%% \includescalefigure{label}{short caption}{long caption}{scale}{filename}
%% - includes a figure with a given label, a short caption for the table of contents and a longer caption that describes the figure in some detail and a scale factor 'scale'
\newcommand{\includescalefigure}[5]{
  \begin{figure}[H]
    \centering
    \includegraphics[width=#4\linewidth]{#5}
    \captionsetup{width=.8\linewidth}
    \caption[#2]{#3}
    \label{#1}
  \end{figure}
}


%%------------------------------------------------
%% Parameters
%%------------------------------------------------
% Set up the header and footer
\newcommand{\assignmentTitle}{Assignment\ \#2: OpenFlow} % Assignment title
\newcommand{\moduleCode}{CS2031}
\newcommand{\moduleName}{Telecommunications\ II}
\newcommand{\authorName}{Lexes\ Jan\ Mantiquilla} % Your name
\newcommand{\authorID}{-1} % Your student ID


\title{\textbf{\moduleCode\ \moduleName\ \assignmentTitle}}
\author{\authorName\ -\ \authorID}
\bibliographystyle{abbrvnat}
% Fix url
\Urlmuskip=0mu plus 1mu
\graphicspath{{images/}}

%%------------------------------------------------
%% Document
%%------------------------------------------------
\begin{document}
\captionsetup{width=.8\linewidth}

\maketitle
\tableofcontents

\newpage

\section{Introduction}

\subsection{Problem Statement}

\subsection{Approach}

\section{Overall Design}

\subsection{Problem Statement}

\subsection{Features}

\subsection{OpenFlow}

\subsection{Design of the Protocol}

\subsection{Packet Descriptions}

\section{Implementation}
This section contains an explanation of the implementation. It will describe how
each of the classes used implement the design described above.

\subsection{Packet Classes}
The packet classes are encoded using the method ObejectInputStream and
ByteArrayInputStream. This allows us to create a byte array which is then used
to create a DatagramPacket. The abstract class PacketContent and the classes
which inherit it abstract the creation of DatagramPackets. The abstract
methods toByteArray and fromDatagramPacket do the encoding of the
DatagramPacket. Each implementation of the PacketContent class has their own
implementation of the abstract methods toByteArray and fromDatagramPacket. The
abstract PacketContent abstract class and the AckPacketContent class have been
provided to us in the Advanced sample code. I have created other packet classes
which implement the PacketContent abstract class. The PacketContent class has
been slightly modified to also store the frame id.

\subsubsection{FeatureRequestPacketContent}

\subsubsection{FeatureResultPacketContent}

\subsubsection{FlowModPacketContent}

\subsubsection{HelloPacketContent}

\subsubsection{PacketContent}

\subsubsection{PacketInPacketContent}

\subsubsection{PayloadPacketContent}

\subsection{Node Classes}

\subsubsection{Controller}

\subsubsection{EndNode}

\subsubsection{Node}

\subsubsection{Switch}

\subsection{Path Finding class}

\subsubsection{Graph}

\subsection{Miscellaneous Classes}

\subsubsection{Terminal}

\subsubsection{Tokenizer}

\section{Advantages and Disadvantages}

\subsection{Advantages}

\subsection{Disadvantages}

\section{Program Usage}

\section{Reflection}

\end{document}
